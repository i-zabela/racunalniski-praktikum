\documentclass[11pt]{article}
\usepackage[slovene]{babel}
\usepackage[a4paper, margin=2.5cm]{geometry}
\usepackage[utf8]{inputenc}
\usepackage[T1]{fontenc}
\usepackage{amsmath}
\usepackage{amsthm}
\usepackage{graphicx}
\usepackage{makeidx}
\usepackage{amsfonts}
\usepackage{hyperref}
\usepackage{booktabs}

{\theoremstyle{definition}
\newtheorem{definicija}{Definicija}
}

{\theoremstyle{plain}
\newtheorem{izrek}{Izrek}
}

\title{Brownovo gibanje}
\date{}
\author{Matej Rojec}


\begin{document}

\maketitle

Brownovo gibanje (več v <\cite{karatzas1991brownian}) je intuitivno slučajen proces, % Sklic na knjigo
ki predstavlja naključno gibanje delcev v mediju.
    

\begin{figure}[h!]
    \centering
    \includegraphics[width=0.4\textwidth]{PerrinPlot2.pdf}
    \caption{Reprodukcija slike iz Jean Baptiste Perrin, \emph{Mouvement brownien et réalité moléculaire}, Ann. de Chimie et de Physique (VIII) 18, 5-114, 1909}
\end{figure}


\begin{definicija}
    Standardno Brownovo gibanje $\{B_t\}_{t \geq 0}$ je slučajen proces z naslednjimi lastnostmi: 
         \begin{enumerate}
            \item $B_0 = 0$.
            \item Prirastki $B_{t_n} - B_{t_{n-1}}, B_{t_{n-1}} - B_{t_{n-2}}, \ldots, B_2 - B_1, B_1 - B_0$ so neodvisne slučajne spremenljivke, za vsak $t_0 \leq t_1 \leq \cdots \leq t_{n-1} \leq t_n$.
            \item Za vsak $t \geq 0$ in $h > 0$ velja $B_{t+h} - B_t \sim \mathcal{N}(0, h)$.
            \item Funkcija $t \mapsto B_t$ je zvezna skoraj gotovo.
         \end{enumerate}
\end{definicija}
    
    Preden zapišemo izrek, definirajmo še pojem časa ustavljanja.
\newcommand{\f}{\mathcal{F}}
    % Začetek definicije
    
    \begin{definicija}
        Slučajna spremenljivka $\tau$ na verjetnostnem prostoru $(\Omega, \f, P)$ z vrednostmi v $\mathbb{R}^+$
        je čas ustavljanja glede na filtracijo $(\f_t)_{t \in T}$, če velja $\forall t \in T:\{\tau \le t\} \in \f_t$.
    \end{definicija}
    % Konec definicije
    
    Zdaj lahko zapišemo izrek ~\ref{thm:stopped_brownian}. % Sklic na izrek z oznako thm:stopped_brownian
    
    % Začetek izreka
   \begin{izrek}
    \label{thm:stopped_brownian}
     Naj bo $\{B_t\}_{t \geq 0}$ (standardno) Brownovo gibanje, $\tau$ čas ustavljanja glede na 
     $(\f_t)_{t \ge 0}$ in naj velja $P[ \tau < \infty ] = 1$.
     Potem je tudi proces:
     \[
     \hat{B} := \{B_{T+t} - B_T \mid t \geq 0\}
     \]
     (standardno) Brownovo gibanje in neodvisen od $\f$.
   \end{izrek}
    % Konec izreka
    

\bibliographystyle{plain}
\bibliography{knjiga.bib}
\printindex
\end{document}